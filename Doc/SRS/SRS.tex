\documentclass[12pt]{article}

\usepackage{amsmath, mathtools}
\usepackage{amsfonts}
\usepackage{amssymb}
\usepackage{graphicx}
\usepackage{colortbl}
\usepackage{xr}
\usepackage{hyperref}
\usepackage{longtable}
\usepackage{xfrac}
\usepackage{tabularx}
\usepackage{float}
\usepackage{siunitx}
\usepackage{booktabs}
\usepackage{caption}
\usepackage{pdflscape}
\usepackage{afterpage}
\usepackage[usenames,dvipsnames,svgnames,table]{xcolor}
\usepackage{txfonts}

\usepackage[round]{natbib}

%\usepackage{refcheck}

\hypersetup{
    bookmarks=true,         % show bookmarks bar?
      colorlinks=true,       % false: boxed links; true: colored links
    linkcolor=Purple,          % color of internal links (change box color with 
    %linkbordercolor)
    citecolor=ForestGreen,        % color of links to bibliography
    filecolor=WildStrawberry,      % color of file links
    urlcolor=Cerulean           % color of external links
}

\input{../Comments}

% For easy change of table widths
\newcommand{\colZwidth}{1.0\textwidth}
\newcommand{\colAwidth}{0.13\textwidth}
\newcommand{\colBwidth}{0.82\textwidth}
\newcommand{\colCwidth}{0.1\textwidth}
\newcommand{\colDwidth}{0.05\textwidth}
\newcommand{\colEwidth}{0.8\textwidth}
\newcommand{\colFwidth}{0.17\textwidth}
\newcommand{\colGwidth}{0.5\textwidth}
\newcommand{\colHwidth}{0.28\textwidth}

% Used so that cross-references have a meaningful prefix
\newcounter{defnum} %Definition Number
\newcommand{\dthedefnum}{GD\thedefnum}
\newcommand{\dref}[1]{GD\ref{#1}}
\newcounter{datadefnum} %Datadefinition Number
\newcommand{\ddthedatadefnum}{DD\thedatadefnum}
\newcommand{\ddref}[1]{DD\ref{#1}}
\newcounter{theorynum} %Theory Number
\newcommand{\tthetheorynum}{T\thetheorynum}
\newcommand{\tref}[1]{T\ref{#1}}
\newcounter{tablenum} %Table Number
\newcommand{\tbthetablenum}{T\thetablenum}
\newcommand{\tbref}[1]{TB\ref{#1}}
\newcounter{assumpnum} %Assumption Number
\newcommand{\atheassumpnum}{P\theassumpnum}
\newcommand{\aref}[1]{A\ref{#1}}
\newcounter{goalnum} %Goal Number
\newcommand{\gthegoalnum}{P\thegoalnum}
\newcommand{\gsref}[1]{GS\ref{#1}}
\newcounter{instnum} %Instance Number
\newcommand{\itheinstnum}{IM\theinstnum}
\newcommand{\iref}[1]{IM\ref{#1}}
\newcounter{reqnum} %Requirement Number
\newcommand{\rthereqnum}{P\thereqnum}
\newcommand{\rref}[1]{R\ref{#1}}
\newcounter{lcnum} %Likely change number
\newcommand{\lthelcnum}{LC\thelcnum}
\newcommand{\lcref}[1]{LC\ref{#1}}

\newcommand{\progname}{Companion Cube Calculator} % PUT YOUR PROGRAM NAME HERE
\newcommand{\prognameAbbrv}{$C^{3}$}

\usepackage{fullpage}

\begin{document}

\title{Companion Cube Calculator} 
\author{Geneva Smith}
\date{\today}
	
\maketitle

\pagenumbering{roman}
\tableofcontents

\begin{table}[bp]
\caption{\bf Revision History}
\begin{tabularx}{\textwidth}{p{3cm}p{2cm}X}
\toprule {\bf Date} & {\bf Version} & {\bf Notes}\\
\midrule
September 26, 2017 & 1.1 & Completed the first draft of the General System 
Description (Section \ref{general})\\
September 25, 2017 & 1.0 & Completed the first draft of the Introduction 
(Section \ref{intro})\\
\bottomrule
\end{tabularx}
\end{table}

\newpage
\section{Reference Material}

This section records information for easy reference.

\subsection{Table of Units}

The tasks performed by the \prognameAbbrv{} tool are unitless.

\subsection{Table of Symbols}

The table that follows summarizes the symbols used in this document along with
their units.  The choice of symbols was made to be consistent with the heat
transfer literature and with existing documentation for solar water heating
systems.  The symbols are listed in alphabetical order.

\renewcommand{\arraystretch}{1.2}
%\noindent \begin{tabularx}{1.0\textwidth}{l l X}
\noindent \begin{longtable*}{l l p{12cm}} \toprule
\textbf{Symbol} & \textbf{Unit} & \textbf{Description}\\
\midrule 
$b^x$ & - & Base number $b$ raised to the power of variable $x$\\
$D(x)$ & - & Domain of $x$\\
$f(X)$ & - & A function of a variable set $X$\\
$R(f(X))$ & - & Range of $f(X)$\\
$\varmathbb{R}$ & - & The set of real numbers\\
$x^n$ & - & Variable $x$ raised to the power of constant $n$ \\
\bottomrule
\end{longtable*}

\subsection{Abbreviations and Acronyms}

\renewcommand{\arraystretch}{1.2}
\begin{tabular}{l l} 
  \toprule		
  \textbf{Text} & \textbf{Description}\\
  \midrule 
  A & Assumption\\
  CRPG & Computer Role-Playing Game\\
  DD & Data Definition\\
  GD & General Definition\\
  GS & Goal Statement\\
  GUI & Graphical User Interface\\
  IM & Instance Model\\
  LC & Likely Change\\
  NPC & Non-Player Character\\
  PS & Physical System Description\\
  R & Requirement\\
  SRS & Software Requirements Specification\\
  \prognameAbbrv{} & \progname{}\\
  T & Theoretical Model\\
  \bottomrule
\end{tabular}

\newpage
\pagenumbering{arabic}

\section{Introduction}
\label{intro}
This document is an SRS for the \progname{} (\prognameAbbrv{}), a mathematical 
tool which determines the range of a user-specified function given the domains 
of the function's variables. This tool is being developed to aid in the 
specification and refinement of GLaDOS, an emotion engine for Non-Player 
Characters (NPCs) in Computer Role-Playing Games (CRPG) as described by 
\citet{glados}.

\subsection{Purpose of Document}
This document outlines the requirements identified for the development of the 
\prognameAbbrv{} tool, including the product goals, product scope, and the 
mathematical models driving the design. It also describes the mathematical 
assumptions, theories, and models used to create the tool. The purpose of 
documenting this information is to aid in future use, maintenance, and 
development of the \prognameAbbrv{} tool.

This document is intended for two reader types -- those who wish to use the 
tool and those who wish to expand the tool. Even though the \progname{} was 
created to aid in the development of a specific system (GLaDOS), the tested 
functions do not have any specific units. This means that it can be used for 
any function that exists in the domain of real numbers ($\varmathbb{R}$). 
Therefore, this document can be used by a user who wishes to use the 
\prognameAbbrv{} tool to determine the range of a specified mathematical 
function. Since the initial development of the \prognameAbbrv{} tool will be 
limited to arithmetic operators, this document includes information that will 
be useful to a developer looking to expand the abilities of the 
\prognameAbbrv{} with additional mathematical models such as trigonometry to 
suit their project.

\subsection{Scope of Requirements} 
The \progname{} calculates the mathematical range of a user-defined function 
using the defined variable domains for that function. In the cases where a 
variable domain is not provided, it will be assumed that the range is 
$(-\infty, \infty)$. For the initial version of this product, the mathematical 
operations allowed in an function will be limited to the four arithmetic 
operators ($+$, $-$, $\times$, $\div$, exponents). Future iterations can expand 
this list to include trigonometric functions ($\sin$, $\cos$, $\tan$, $\arcsin$,
$\arccos$, $\arctan$), partial functions, and other useful operations and 
functions.

The purpose of this product is to aid in the design and tuning of the GLaDOS 
architecture, a specialized game engine enables game designers to create 
Non-Player Characters (NPCs) that react to their environment by using models of 
emotion from psychology. One module in the architecture converts information 
from the environment into an internal representation that directs an NPC's 
decision-making. This task requires the specification of numerical functions 
with multiple variables which must be normalized to a range of $[-1,1]$. In 
order to normalize the engine's calculation, the range of the function must be 
known. Currently, the encoded functions are not well-informed by observation or 
scientific research and must be subjected to an iterative process in order to 
address this problem. An automated method of calculating the range of a 
proposed function will make this process faster and less error-prone.

Although this product is being designed for a specific project, the concepts 
involved can be used in similar projects because it does not contain any 
project-specific concepts or models.

\subsection{Characteristics of Intended Reader}
\label{intro_reader}
The intended reader of this document must understand elementary algebra, 
especially notation and functions, in the domain of real numbers 
($\varmathbb{R}$). They must also have an understanding of domain and range 
with respect to a function in order to understand the outputs of the product 
and how it relates to its 
inputs.

\subsection{Organization of Document}
This document begins by describing the general description of the \progname{} 
tool (Section \ref{general}), which includes the system context, constraints, 
and the intended users' characteristics. This is followed by a description of 
the problem to be solved, including the models and assumptions that are used to 
solve it (Section \ref{specific}). This description is followed by the 
functional and non-functional requirements of the \prognameAbbrv{} tool 
(Section \ref{requirements}) and the potential changes that will be made to 
them in future iterations (Section \ref{changes}). The last section in this 
document, traceability matrices and graphs, visually describes the dependencies 
between different document components (Section \ref{trace}). This information 
should be used by making changes to the decisions made for this version of the 
tool.

\section{General System Description}
\label{general}

This section identifies the interfaces between the system and its environment,
describes the user characteristics and lists the system constraints.

\subsection{System Context}
The \progname{} tool is a stand-alone application for calculating the range of 
a user-provided function given the known domains of the function's component 
variables. Therefore it is independent and self-contained with respect to 
external organizations, products, and technologies.

The user interface of the \prognameAbbrv{} tool facilitates communication 
between the product and the user, and must contain the ability to exchange user 
inputs and system outputs. In general, the user is responsible for ensuring 
that they have provided a semantically correct function and that their inputs 
do not contain mathematical operations that are not implemented in the tool. 
The \progname{} is responsible for providing an information exchange interface 
between itself and the user, performing mathematical calculations, and 
detecting syntactic errors in the system inputs.

\begin{itemize}
	\item User Responsibilities:
	\begin{itemize}
		\item Provide an function from the domain of real numbers 
		($\varmathbb{R}$) that only contains mathematical operations that are 
		recognized by the system
		\item Provide the mathematical domains of the function's variables if 
		they are known
		\item Ensure that the function given to the tool does not contain 
		semantic errors
		\item Ensure that the products outputs are semantically correct with 
		respect to the target application
	\end{itemize}
	\item \progname{} Responsibilities:
	\begin{itemize}
		\item Allow the user to enter their function and known variable domains 
		as system inputs
		\item For variables with unknown domains, provide default values to be 
		used in calculations
		\item Detect data type mismatch, such as a string of characters instead 
		of a floating point number
		\item Detect bracket mismatches
		\item Calculate and output the mathematical range corresponding to the 
		user's inputs; if a result cannot be calculated, communicate this to 
		the user
	\end{itemize}
\end{itemize}

\subsection{User Characteristics} \label{SecUserCharacteristics}
The end user of \progname{} is also the intended reader (Section 
\ref{intro_reader}) of this document.

\subsection{System Constraints}
The use of a Graphical User Interface (GUI) is useful for the \prognameAbbrv{} 
tool because it can help the user visualize their function during design time 
and debugging. This has implications on the types of languages that this tool 
will be developed in and target platform choices. If a GUI is included in the 
\prognameAbbrv{} design, only Windows platforms will be supported in the 
initial release.

\section{Specific System Description}
\label{specific}
This section first presents the problem description, which gives a high-level
view of the problem to be solved.  This is followed by the solution 
characteristics specification, which presents the assumptions, theories, 
definitions and finally the instance models.  \wss{Add any project specific 
details that are relevant for the section overview.}

\subsection{Problem Description} \label{Sec_pd}
The GLaDOS architecture is a specialized game engine that enables game 
designers to create Non-Player Characters (NPCs) that react to their 
environment by using models of emotion from psychology. A key component of the 
architecture is the primary appraisal module, where inputs from the environment 
are converted into emotion values. The functions currently used for this task 
are not well informed by any scientific research, and will be incrementally 
revised as more information is collected. These new functions are required to 
conform to the same mathematical range as the currently implemented functions. 
Ensuring that this requirement is maintained over consecutive iterations of the 
functions will become tedious and time consuming, distracting from the main 
task of developing more rigorous and explainable emotion functions.

The \progname{} is a tool for calculating the range of a mathematical function 
based on its input variables, which will help make the function design process 
faster and less error-prone when compared to doing it by hand.

\subsubsection{Terminology and  Definitions}

This subsection provides a list of terms that are used in the subsequent
sections and their meaning, with the purpose of reducing ambiguity and making it
easier to correctly understand the requirements:

\begin{itemize}

\item Closed interval: A bounded set that includes its end points; this is 
expressed $[a,b]$
\item Connected set: a set of values that does not contain any disjoint members
\item Domain ($D(x)$): the set of values for a variable $x$ that are valid in 
$f(X)$
\item Extended real interval: the set of all real numbers that also contains 
$\pm \infty$
\item Open interval: A bounded set of values that does not contain its end 
points; this is expressed $(a,b)$
\item Range ($R(f(X))$): the set of values that are produced by $f(X)$
\item Real interval: a closed, connected set of real numbers

\end{itemize}

\subsubsection{Physical System Description}

The \progname{} tool does not have a physical system component because it 
exists independently of the context of $f(X)$. 

\subsubsection{Goal Statements}

\noindent Given an function and a set of variables, the goal statements are:

\begin{itemize}

\item[GS\refstepcounter{goalnum}\thegoalnum \label{G_range}:] 
Calculate $R(f(X))$.

\item[GS\refstepcounter{goalnum}\thegoalnum \label{G_unknownDomain}:] 
For any $x \in X$, calculate $D(x)$ if it is not part of the input set.

\end{itemize}

\subsection{Solution Characteristics Specification}
The instance models that govern \progname{} are presented in
Subsection~\ref{sec_instance}. The information to understand the meaning of the
instance models and their derivation is also presented, so that the instance
models can be verified.

\subsubsection{Assumptions}
This section simplifies the original problem and helps in developing the
theoretical model by filling in the missing information for the physical
system. The numbers given in the square brackets refer to the theoretical model
[T], general definition [GD], data definition [DD], instance model [IM], or
likely change [LC], in which the respective assumption is used.

\begin{itemize}

\item[A\refstepcounter{assumpnum}\theassumpnum \label{A_domain}:] $\forall x 
\in X | D(x) \in \varmathbb{R}$.

\item[A\refstepcounter{assumpnum}\theassumpnum \label{A_interval}:] $R(f(X))$ 
is a real interval.

\item[A\refstepcounter{assumpnum}\theassumpnum \label{A_operators}:] The 
mathematical operators used in $f(X)$ can be found in the set $\{+, -, 
\times, \div, b^x, x^n \}$.

\end{itemize}

\subsubsection{Theoretical Models}\label{sec_theoretical}

This section focuses on the general equations and laws that \progname{} is based
on.  \wss{Modify the examples below for your problem, and add additional models
  as appropriate.}

~\newline

\noindent
\begin{minipage}{\textwidth}
\renewcommand*{\arraystretch}{1.5}
\begin{tabular}{| p{\colAwidth} | p{\colBwidth}|}
  \hline
  \rowcolor[gray]{0.9}
  Number& T\refstepcounter{theorynum}\thetheorynum \label{T_addition}\\
  \hline
  Label&\bf Addition on Real Intervals\\
  \hline
  Equation&  $X + Y = \{x + y | x \in X \wedge y \in Y\}$\\
  \hline
  Description & \\
  \hline
  Source & \citet{intervalarithmetic}\\
  \hline
  Ref.\ By & \dref{ROCT}\\
  \hline
\end{tabular}
\end{minipage}\\

~\newline

\noindent
\begin{minipage}{\textwidth}
	\renewcommand*{\arraystretch}{1.5}
	\begin{tabular}{| p{\colAwidth} | p{\colBwidth}|}
		\hline
		\rowcolor[gray]{0.9}
		Number& T\refstepcounter{theorynum}\thetheorynum \label{T_subtraction}\\
		\hline
		Label&\bf Subtraction on Real Intervals\\
		\hline
		Equation&  $X - Y = \{x - y | x \in X \wedge y \in Y\}$\\
		\hline
		Description & \\
		\hline
		Source & \citet{intervalarithmetic}\\
		\hline
		Ref.\ By & \dref{ROCT}\\
		\hline
	\end{tabular}
\end{minipage}\\

~\newline

\noindent
\begin{minipage}{\textwidth}
	\renewcommand*{\arraystretch}{1.5}
	\begin{tabular}{| p{\colAwidth} | p{\colBwidth}|}
		\hline
		\rowcolor[gray]{0.9}
		Number& T\refstepcounter{theorynum}\thetheorynum 
		\label{T_multiplication}\\
		\hline
		Label&\bf Multiplication on Real Intervals\\
		\hline
		Equation&  $X \times Y = \{x \times y | x \in X \wedge y \in Y\}$\\
		\hline
		Description & \\
		\hline
		Source & \citet{intervalarithmetic}\\
		\hline
		Ref.\ By & \dref{ROCT}\\
		\hline
	\end{tabular}
\end{minipage}\\

~\newline

\noindent
\begin{minipage}{\textwidth}
	\renewcommand*{\arraystretch}{1.5}
	\begin{tabular}{| p{\colAwidth} | p{\colBwidth}|}
		\hline
		\rowcolor[gray]{0.9}
		Number& T\refstepcounter{theorynum}\thetheorynum 
		\label{T_division}\\
		\hline
		Label&\bf Division on Real Intervals\\
		\hline
		Equation&  $X \div Y = \{z | \exists x \in X, y \in Y \wedge y \neq 0, 
		z 
		= x \div y\}$\\
		\hline
		Description & \\
		\hline
		Source & \citet{intervalarithmetic}\\
		\hline
		Ref.\ By & \dref{ROCT}\\
		\hline
	\end{tabular}
\end{minipage}\\

~\newline

\noindent
\begin{minipage}{\textwidth}
	\renewcommand*{\arraystretch}{1.5}
	\begin{tabular}{| p{\colAwidth} | p{\colBwidth}|}
		\hline
		\rowcolor[gray]{0.9}
		Number& T\refstepcounter{theorynum}\thetheorynum 
		\label{T_exponents}\\
		\hline
		Label&\bf Real Interval Exponents on a Constant Base Number\\
		\hline
		Equation&  $b^X = \{b^x | x \in X \wedge b > 1\}$\\
		\hline
		Description & \\
		\hline
		Source & \url{https://en.wikipedia.org/wiki/Interval_arithmetic}\\
		\hline
		Ref.\ By & \dref{ROCT}\\
		\hline
	\end{tabular}
\end{minipage}\\

~\newline

\noindent
\begin{minipage}{\textwidth}
	\renewcommand*{\arraystretch}{1.5}
	\begin{tabular}{| p{\colAwidth} | p{\colBwidth}|}
		\hline
		\rowcolor[gray]{0.9}
		Number& T\refstepcounter{theorynum}\thetheorynum 
		\label{T_expbase}\\
		\hline
		Label&\bf Constant Exponents on a Real Interval Base 
		Number\\
		\hline
		Equations &  $X^n = \begin{cases}
			\{x^n | x \in X \wedge n 
			\in \varmathbb{N}\} & \text{n is odd}\\
			\{[x_{1}^n,...,x_{z}^n] | x \in X \wedge n 
			\in \varmathbb{N}\} & \text{n is even, }x_{z} \geq x_{z+1}\\
			\{[x_{z}^n,...,x_{1}^n] | x \in X \wedge n 
			\in \varmathbb{N}\} & \text{n is even, }x_{z} < x_{z+1}\\
			\{[0,...,max\{x_{1}^n,x_{z}^n\}] | x \in X \wedge n 
			\in \varmathbb{N}\} & \text{ELSE}\\
		\end{cases}$
		\newline
		\\
		\hline
		Description & \\
		\hline
		Source & \url{https://en.wikipedia.org/wiki/Interval_arithmetic}\\
		\hline
		Ref.\ By & \dref{ROCT}\\
		\hline
	\end{tabular}
\end{minipage}\\

~\newline

\subsubsection{General Definitions}\label{sec_gendef}

This section collects the laws and equations that will be used in deriving the
data definitions, which in turn are used to build the instance models.
\wss{Some projects may not have any content for this section, but the section
  heading should be kept.}  \wss{Modify the examples below for your problem, and
  add additional definitions as appropriate.}

~\newline

\noindent
\begin{minipage}{\textwidth}
\renewcommand*{\arraystretch}{1.5}
\begin{tabular}{| p{\colAwidth} | p{\colBwidth}|}
\hline
\rowcolor[gray]{0.9}
Number& GD\refstepcounter{defnum}\thedefnum \label{NL}\\
\hline
Label &\bf Newton's law of cooling \\
\hline
% Units&$MLt^{-3}T^0$\\
% \hline
SI Units&\si{\watt\per\square\metre}\\
\hline
Equation&$ q(t) = h \Delta T(t)$  \\
\hline
Description &
Newton's law of cooling describes convective cooling from a surface.  The law is
stated as: the rate of heat loss from a body is proportional to the difference
in temperatures between the body and its surroundings.
\\
& $q(t)$ is the thermal flux (\si{\watt\per\square\metre}).\\
& $h$ is the heat transfer coefficient, assumed independent of $T$ (\aref{A_hcoeff})
	(\si{\watt\per\square\metre\per\celsius}).\\
&$\Delta T(t)$= $T(t) - T_{\text{env}}(t)$ is the time-dependent thermal gradient
between the environment and the object (\si{\celsius}).
\\
\hline
  Source &~\cite[p.\ 8]{Incropera2007}\\
  \hline
  Ref.\ By & \ddref{FluxCoil}, \ddref{FluxPCM}\\
  \hline
\end{tabular}
\end{minipage}\\

\subsubsection*{Detailed derivation of simplified rate of change of temperature}

\wss{This may be necessary when the necessary information does not fit in the
  description field.} 

\subsubsection{Data Definitions}\label{sec_datadef}

This section collects and defines all the data needed to build the instance
models. The dimension of each quantity is also given.  \wss{Modify the examples
  below for your problem, and add additional definitions as appropriate.}

~\newline

\noindent
\begin{minipage}{\textwidth}
\renewcommand*{\arraystretch}{1.5}
\begin{tabular}{| p{\colAwidth} | p{\colBwidth}|}
\hline
\rowcolor[gray]{0.9}
Number& DD\refstepcounter{datadefnum}\thedatadefnum \label{FluxCoil}\\
\hline
Label& \bf Heat flux out of coil\\
\hline
Symbol &$q_C$\\
\hline
% Units& $Mt^{-3}$\\
% \hline
  SI Units & \si{\watt\per\square\metre}\\
  \hline
  Equation&$q_C(t) = h_C (T_C - T_W(t))$, over area $A_C$\\
  \hline
  Description & 
                $T_C$ is the temperature of the coil (\si{\celsius}).  $T_W$ is the temperature of the water (\si{\celsius}).  
                The heat flux out of the coil, $q_C$ (\si{\watt\per\square\metre}), is found by
                assuming that Newton's Law 
                of Cooling applies (\aref{A_Newt_coil}).  This law (\dref{NL}) is used on the surface of
                the coil, which has area $A_C$ (\si{\square\metre}) and heat 
                transfer coefficient $h_C$
                (\si{\watt\per\square\metre\per\celsius}).  This equation
                assumes that the temperature of the coil is constant over time (\aref{A_tcoil}) and that it does not vary along the length
                of the coil (\aref{A_tlcoil}).
  \\
  \hline
  Sources&~\cite{Lightstone2012}  \\
  \hline
  Ref.\ By & \iref{ewat}\\
  \hline
\end{tabular}
\end{minipage}\\

\subsubsection{Instance Models} \label{sec_instance}    

This section transforms the problem defined in Section~\ref{Sec_pd} into 
one which is expressed in mathematical terms. It uses concrete symbols defined 
in Section~\ref{sec_datadef} to replace the abstract symbols in the models 
identified in Sections~\ref{sec_theoretical} and~\ref{sec_gendef}.

The goals \wss{reference your goals} are solved by \wss{reference your instance
  models}.  \wss{other details, with cross-references where appropriate.}
\wss{Modify the examples below for your problem, and add additional models as
  appropriate.}

~\newline

%Instance Model 1

\noindent
\begin{minipage}{\textwidth}
\renewcommand*{\arraystretch}{1.5}
\begin{tabular}{| p{\colAwidth} | p{\colBwidth}|}
  \hline
  \rowcolor[gray]{0.9}
  Number& IM\refstepcounter{instnum}\theinstnum \label{ewat}\\
  \hline
  Label& \bf Energy balance on water to find $T_W$\\
  \hline
  Input&$m_W$, $C_W$, $h_C$, $A_C$, $h_P$, $A_P$, $t_\text{final}$, $T_C$, 
  $T_\text{init}$, $T_P(t)$ from \iref{epcm}\\
  & The input is constrained so that $T_\text{init} \leq T_C$ (\aref{A_charge})\\
  \hline
  Output&$T_W(t)$, $0\leq t \leq t_\text{final}$, such that\\
  &$\frac{dT_W}{dt} = \frac{1}{\tau_W}[(T_C - T_W(t)) + {\eta}(T_P(t) - T_W(t))]$,\\
  &$T_W(0) = T_P(0) = T_\text{init}$ (\aref{A_InitTemp}) and $T_P(t)$ from \iref{epcm} \\
  \hline
  Description&$T_W$ is the water temperature (\si{\celsius}).\\
  &$T_P$ is the PCM temperature (\si{\celsius}).\\
  &$T_C$ is the coil temperature (\si{\celsius}).\\
  &$\tau_W = \frac{m_W C_W}{h_C A_C}$ is a constant (\si{\second}).\\
  &$\eta = \frac{h_P A_P}{h_C A_C}$ is a constant (dimensionless).\\
  & The above equation applies as long as the water is in liquid form,
  $0<T_W<100^o\text{C}$, where $0^o\text{C}$ and $100^o\text{C}$ are the melting
  and boiling points of water, respectively (\aref{A_OpRange}, \aref{A_Pressure}).
  \\
  \hline
  Sources&~\cite{Lightstone2012} \ \\
  \hline
  Ref.\ By & \iref{epcm}\\
  \hline
\end{tabular}
\end{minipage}\\

%~\newline

\subsubsection*{Derivation of ...}

\wss{May be necessary to include this subsection in some cases.}

\subsubsection{Data Constraints} \label{sec_DataConstraints}    

Tables~\ref{TblInputVar} and \ref{TblOutputVar} show the data constraints on the
input and output variables, respectively.  The column for physical constraints gives
the physical limitations on the range of values that can be taken by the
variable.  The column for software constraints restricts the range of inputs to
reasonable values.  The constraints are conservative, to give the user of the
model the flexibility to experiment with unusual situations.  The column of
typical values is intended to provide a feel for a common scenario.  The
uncertainty column provides an estimate of the confidence with which the
physical quantities can be measured.  This information would be part of the
input if one were performing an uncertainty quantification exercise.

The specification parameters in Table~\ref{TblInputVar} are listed in
Table~\ref{TblSpecParams}.

\begin{table}[!h]
  \caption{Input Variables} \label{TblInputVar}
  \renewcommand{\arraystretch}{1.2}
\noindent \begin{longtable*}{l l l l c} 
  \toprule
  \textbf{Var} & \textbf{Physical Constraints} & \textbf{Software Constraints} &
                             \textbf{Typical Value} & \textbf{Uncertainty}\\
  \midrule 
  $L$ & $L > 0$ & $L_{\text{min}} \leq L \leq L_{\text{max}}$ & 1.5 \si[per-mode=symbol] {\metre} & 10\%
  \\
  \bottomrule
\end{longtable*}
\end{table}

\noindent 
\begin{description}
\item[(*)] \wss{you might need to add some notes or clarifications}
\end{description}

\begin{table}[!h]
\caption{Specification Parameter Values} \label{TblSpecParams}
\renewcommand{\arraystretch}{1.2}
\noindent \begin{longtable*}{l l} 
  \toprule
  \textbf{Var} & \textbf{Value} \\
  \midrule 
  $L_\text{min}$ & 0.1 \si{\metre}\\
  \bottomrule
\end{longtable*}
\end{table}

\begin{table}[!h]
\caption{Output Variables} \label{TblOutputVar}
\renewcommand{\arraystretch}{1.2}
\noindent \begin{longtable*}{l l} 
  \toprule
  \textbf{Var} & \textbf{Physical Constraints} \\
  \midrule 
  $T_W$ & $T_\text{init} \leq T_W \leq T_C$ (by~\aref{A_charge})
  \\
  \bottomrule
\end{longtable*}
\end{table}

\subsubsection{Properties of a Correct Solution} \label{sec_CorrectSolution}

\noindent
A correct solution must exhibit \wss{fill in the details}

\section{Requirements}
\label{requirements}

This section provides the functional requirements, the business tasks that the
software is expected to complete, and the nonfunctional requirements, the
qualities that the software is expected to exhibit.

\subsection{Functional Requirements}

\noindent \begin{itemize}

\item[R\refstepcounter{reqnum}\thereqnum \label{R_Inputs}:] \wss{Requirements
    for the inputs that are supplied by the user.  This information has to be
    explicit.}

\item[R\refstepcounter{reqnum}\thereqnum \label{R_OutputInputs}:] \wss{It isn't
    always required, but often echoing the inputs as part of the output is a
    good idea.}

\item[R\refstepcounter{reqnum}\thereqnum \label{R_Calculate}:] \wss{Calculation
    related requirements.}

\item[R\refstepcounter{reqnum}\thereqnum \label{R_VerifyOutput}:]
  \wss{Verification related requirements.}

\item[R\refstepcounter{reqnum}\thereqnum \label{R_Output}:] \wss{Output related
    requirements.}

\end{itemize}

\subsection{Nonfunctional Requirements}

\wss{List your nonfunctional requirements.  You may consider using a fit
  criterion to make them verifiable.}

\section{Likely Changes}    
\label{changes}

\noindent \begin{itemize}

\item[LC\refstepcounter{lcnum}\thelcnum\label{LC_meaningfulLabel}:] \wss{Give
    the likely changes, with a reference to the related assumption (aref), as appropriate.}

\end{itemize}

\section{Traceability Matrices and Graphs}
\label{trace}

The purpose of the traceability matrices is to provide easy references on what
has to be additionally modified if a certain component is changed.  Every time a
component is changed, the items in the column of that component that are marked
with an ``X'' may have to be modified as well.  Table~\ref{Table:trace} shows the
dependencies of theoretical models, general definitions, data definitions, and
instance models with each other. Table~\ref{Table:R_trace} shows the
dependencies of instance models, requirements, and data constraints on each
other. Table~\ref{Table:A_trace} shows the dependencies of theoretical models,
general definitions, data definitions, instance models, and likely changes on
the assumptions.

\wss{You will have to modify these tables for your problem.}

\afterpage{
\begin{landscape}
\begin{table}[h!]
\centering
\begin{tabular}{|c|c|c|c|}
\hline
	& \aref{A_domain} & \aref{A_interval} & \aref{A_operators} 
	\\
\hline
\tref{T_addition} & X & X & \\ \hline
\tref{T_subtraction} & X & X & \\ \hline
\tref{T_multiplication} & X & X & \\ \hline
\tref{T_division} & X & X & \\ \hline
\tref{T_exponents} & X & X & \\ \hline
\tref{T_expbase} & X & X & \\ \hline
\dref{NL} & & & \\ \hline
\dref{ROCT} & & & \\ \hline
\ddref{FluxCoil} & & &  \\ \hline
\ddref{FluxPCM} & & &  \\ \hline
\ddref{D_HOF} & & &  \\ \hline
\ddref{D_MF} & & &  \\ \hline
\iref{ewat}         & & &  \\ \hline
\iref{epcm}         & & &  \\ \hline
\iref{I_HWAT}       & & &  \\ \hline
\iref{I_HPCM}       & & &  \\ \hline
\lcref{LC_tpcm}     & & &  \\ \hline
\lcref{LC_tcoil}    & & &  \\ \hline
\lcref{LC_tlcoil}   & & &  \\ \hline
\lcref{LC_charge}   & & &  \\ \hline
\lcref{LC_InitTemp} & & &  \\ \hline
\lcref{LC_htank}    & & &  \\
\hline
\end{tabular}
\caption{Traceability Matrix Showing the Connections Between Assumptions and Other Items}
\label{Table:A_trace}
\end{table}
\end{landscape}
}

\begin{table}[h!]
\centering
\begin{tabular}{|c|c|c|c|c|c|c|c|c|c|c|c|c|c|c|c|c|c|c|c|c|c|c|c|}
\hline        
	& \tref{T_addition}& \tref{T_subtraction}& \tref{T_multiplication}& 
	\tref{T_division}& \tref{T_exponents} & \tref{T_expbase}& \ddref{FluxPCM} & 
	\ddref{D_HOF}& \ddref{D_MF}& \iref{ewat}& \iref{epcm}& \iref{I_HWAT}& 
	\iref{I_HPCM} \\
\hline
\tref{T_addition} & & & & & & & & & & & & & \\ \hline
\tref{T_subtraction} & & & X& & & & & & & & & & \\ \hline
\tref{T_multiplication} & & & & & & & & & & & & & \\ \hline
\tref{T_division} & & & & & & & & & & & & & \\ \hline
\tref{T_exponents} & X& & & & & & & & & & & & \\ \hline
\tref{T_expbase} & & & & X& & & & & & & & & \\ \hline
\ddref{FluxPCM}  & & & & X& & & & & & & & & \\ \hline
\ddref{D_HOF}    & & & & & & & & & & & & & \\ \hline
\ddref{D_MF}     & & & & & & & & X& & & & & \\ \hline
\iref{ewat}      & & & & & X& X& X& & & & X& & \\ \hline
\iref{epcm}      & & & & & X& & X& & X& X& & & X \\ \hline
\iref{I_HWAT}    & & X& & & & & & & & & & & \\ \hline
\iref{I_HPCM}    & & X& X& & & & X& X& X& & X& & \\
\hline
\end{tabular}
\caption{Traceability Matrix Showing the Connections Between Items of Different Sections}
\label{Table:trace}
\end{table}

\begin{table}[h!]
\centering
\begin{tabular}{|c|c|c|c|c|c|c|c|}
\hline
	& \iref{ewat}& \iref{epcm}& \iref{I_HWAT}& \iref{I_HPCM}& \ref{sec_DataConstraints}& \rref{R_RawInputs}& \rref{R_MassInputs} \\
\hline
\iref{ewat}            & & X& & & & X& X \\ \hline
\iref{epcm}            & X& & & X& & X& X \\ \hline
\iref{I_HWAT}          & & & & & & X& X \\ \hline
\iref{I_HPCM}          & & X& & & & X& X \\ \hline
\rref{R_RawInputs}     & & & & & & & \\ \hline
\rref{R_MassInputs}    & & & & & & X& \\ \hline
\rref{R_CheckInputs}   & & & & & X& & \\ \hline
\rref{R_OutputInputs}  & X& X& & & & X& X \\ \hline
\rref{R_TempWater}     & X& & & & & & \\ \hline 
\rref{R_TempPCM}       & & X& & & & & \\ \hline
\rref{R_EnergyWater}   & & & X& & & & \\ \hline
\rref{R_EnergyPCM}     & & & & X& & & \\ \hline
\rref{R_VerifyOutput}  & & & X& X& & & \\ \hline
\rref{R_timeMeltBegin} & & X& & & & & \\ \hline
\rref{R_timeMeltEnd}   & & X& & & & & \\ 
\hline
\end{tabular}
\caption{Traceability Matrix Showing the Connections Between Requirements and Instance Models}
\label{Table:R_trace}
\end{table}

The purpose of the traceability graphs is also to provide easy references on
what has to be additionally modified if a certain component is changed.  The
arrows in the graphs represent dependencies. The component at the tail of an
arrow is depended on by the component at the head of that arrow. Therefore, if a
component is changed, the components that it points to should also be
changed. Figure~\ref{Fig_ATrace} shows the dependencies of theoretical models,
general definitions, data definitions, instance models, likely changes, and
assumptions on each other. Figure~\ref{Fig_RTrace} shows the dependencies of
instance models, requirements, and data constraints on each other.

% \begin{figure}[h!]
% 	\begin{center}
% 		%\rotatebox{-90}
% 		{
% 			\includegraphics[width=\textwidth]{ATrace.png}
% 		}
% 		\caption{\label{Fig_ATrace} Traceability Matrix Showing the Connections Between Items of Different Sections}
% 	\end{center}
% \end{figure}


% \begin{figure}[h!]
% 	\begin{center}
% 		%\rotatebox{-90}
% 		{
% 			\includegraphics[width=0.7\textwidth]{RTrace.png}
% 		}
% 		\caption{\label{Fig_RTrace} Traceability Matrix Showing the Connections Between Requirements, Instance Models, and Data Constraints}
% 	\end{center}
% \end{figure}

\newpage

\bibliographystyle {plainnat}
\bibliography 
{../../ReferenceMaterial/SRS_Refs,../../ReferenceMaterial/ProblemStatement_Refs}

\newpage

\section{Appendix}

\wss{Your report may require an appendix.  For instance, this is a good point to
show the values of the symbolic parameters introduced in the report.}

\subsection{Symbolic Parameters}

\renewcommand{\arraystretch}{1.2}
\begin{tabular}{l l} 		
	$\varmathbb{R}$ & Domain of real numbers\\
\end{tabular}\\

\end{document}