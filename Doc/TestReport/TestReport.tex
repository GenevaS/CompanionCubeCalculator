\documentclass[12pt, titlepage]{article}

\usepackage{longtable}
\usepackage{booktabs}
\usepackage{tabularx}
\usepackage[usenames,dvipsnames,table]{xcolor}
\usepackage{hyperref}
\hypersetup{
	colorlinks,
	citecolor=ForestGreen,
	filecolor=WildStrawberry,
	linkcolor=Purple,
	urlcolor=Cerulean
}
\usepackage[round]{natbib}
\usepackage{float}

\usepackage{xr}
\externaldocument{../SRS/SRS}
\newcommand{\rref}[1]{R\ref{#1}}

\externaldocument{../Design/MIS/MIS}
\newcommand{\misref}[1]{MIS: \ref{#1}}

\input{../Comments}

\begin{document}

\title{Test Report: Companion Cube Calculator} 
\author{Geneva Smith}
\date{\today}
	
\maketitle

\pagenumbering{roman}

\section{Revision History}

\begin{tabularx}{\textwidth}{p{3cm}p{2cm}X}
\toprule {\bf Date} & {\bf Version} & {\bf Notes}\\
\midrule
Date 1 & 1.0 & Notes\\
Date 2 & 1.1 & Notes\\
\bottomrule
\end{tabularx}

~\newpage

\section{Symbols, Abbreviations and Acronyms}

\renewcommand{\arraystretch}{1.2}
\begin{tabular}{l l} 
  \toprule		
  \textbf{symbol} & \textbf{description}\\
  \midrule 
  R & Requirement \\
  T & Test\\
  \bottomrule
\end{tabular}\\

\newpage

\tableofcontents

\listoftables %if appropriate

\listoffigures %if appropriate

\newpage

\pagenumbering{arabic}

\section{Introduction}
This is the test report for the Companion Cube Calculator, a mathematical 
tool which determines the range of a user-specified function given the domains 
of the function's variables. The the directory for this project can be found at:

\begin{center}
	\href{https://github.com/GenevaS/CAS741}{https://github.com/GenevaS/CAS741}.
\end{center} 

\section{Functional Requirements Evaluation}

The following tests have failed by verification:

	\begin{center}
		\begin{longtable}{ | p{3cm} | p{3cm} | p{2cm} | p{2cm} | p{3cm} |}
			\caption{Failed Functional Test Summary} \\ \hline 
			\label{TblInputVar} 
			ID & Input & Expected Outcome & Expected MsgID & Actual MsgID \\ 
			\hline
			test-control\_ precedenceOfOperators3 & 
			\texttt{"x\textasciicircum2*y", 
				"x,2,4\textbackslash ny,3,5"}, 
				\texttt{"(x\textasciicircum2)*y", 
				"x,2,4\textbackslash ny,3,5"} & $TRUE$ & - & (EQC\_ 
				INCOMPLETE\_OP) 
			Error: Unrecognized sequence encountered during Atomic Equation 
			parsing. Remaining equation = \texttt{)*y}. \\ \hline
			
			test-control\_ precedenceOfOperators6 & 
			\texttt{"(2(x+y)\textasciicircum2) /(3\textasciicircum z)", 
				"x,1,2\textbackslash ny,3,4 \textbackslash nz,5,6"} & 
			\texttt{"[0.0438 9574759945 13, \textbackslash n 0.2962962 
			96296296]"} 
			& 
			Range calculated successfully. & (EQC\_ INCOMPLETE\_OP) Error: 
			Unrecognized sequence 
			encountered during Atomic Equation parsing. Remaining equation = 
			\texttt{)/(3\textasciicircum z)}. \\ \hline
		\end{longtable}
	\end{center}


\section{Non-Functional Requirements Evaluation}


\subsection{Correctness}
Correctness testing could not be completed in this project cycle due to time 
constraints.

These tests are directly related to \rref{R_conditionfx} and 
\rref{R_CalculateCompose} (Decomposing the user equation into components and 
recomposing the results).

\subsection{Robustness}
The robustness requirement for recognizing violated data constraints was 
covered in the functional tests:

\begin{itemize}
	\item The constraints on supported operators are contained in the Range 
	Solver test suite. The containment of all operator-specific information 
	within this module made it possible to collect all of these restrictions in 
	the same suite. This design also implicitly supported the output constraint 
	on $R(f(V))$ because only mathematical operations that produced closed, 
	real intervals were implemented.
	\item The constraint of having every $D(v) \in V$ defined as a closed, real 
	interval are contained in the Interval Conversion and Interval Data 
	Structure modules.
\end{itemize}

These tests are directly related to \rref{R_verifyinputs} and 
\rref{R_VerifyOutputConstraints} (Verifying that the program satisfies the 
input and output constraints).

\subsection{Verifiability}
The verifiability requirement stated that the program must be created in a way 
in which its calculations can be checked for correctness. By basing this design 
on verifiable mathematical concepts and implementing the equation decomposition 
using a grammar definition, it is possible to measure if this requirement has 
met. However, verifiability testing could not be completed in this project 
cycle due to time constraints.

This is indirectly related to \rref{R_Output} because the outputs must be shown 
to the program user such that they understand and have confidence in the 
program's results.

\subsection{Usability}
		
\subsection{Maintainability}
The maintainability requirements focus on the extensibility of the original 
implementation with respect to its supported mathematical operations. Support 
for open, real intervals already exists in the Interval Conversion and Interval 
Data Structure modules. This means that it is possible that only the Range 
Solver module would need to be updated to add more mathematical operations. 
However, maintainability testing could not be completed in this project cycle 
due to time constraints.

\section{Unit Testing}
In addition to the functional requirements, unit tests were implemented to 
achieve 100\% code coverage. The purpose of this was to ensure that all code 
paths were being executed and to help identify program errors that were not 
covered in the functional testing suite. Implementation files that were 
automatically generated or that were added to implement the GUI are not covered 
in the unit tests.


\begin{center}
		\begin{longtable}{ | p{2.5cm} | p{4cm} | p{2.5cm} | m{1cm} | m{1.5cm} 
		|}
		\caption{Unit Test Summary} \\ \hline \label{TblUnitTests} 
		\textbf{Test Suite} & \textbf{Test File} & \textbf{Target Modules} & 
		\textbf{Total Tests} & 
		\textbf{Tests Passing (\%)}  \\ 
		\hline
		Control Flow & \texttt{ControlTests.cs} & ControlFlow 
		(\misref{Module_controlflow}) & 6 & 100\% \\ 
		\hline
		
		User Input & \texttt{InputTests.cs} &Input (\misref{Module_userinput}) 
		& 10 & 100\% \\ \hline
		
		Interval & \texttt{IntervalTests.cs} &Interval Data Structure 
		(\misref{Module_intervaldatastructure}), Interval Conversion 
		(\misref{Module_intervalconversion}) & 7 & 100\% \\ \hline
		
		Equation & \texttt{EquationTests.cs} &Equation Data Structure 
		(\misref{Module_equationdatastructure}), Equation Conversion 
		(\misref{Module_equationconversion}) & 16 & 100\% \\ \hline
		
		Variable Consolidation & \texttt{VariableConsoli dationTests.cs} & 
		Consolidate 
		(\misref{Module_variableconsolidation}) & 8 & 100\% \\ \hline
		
		Solver &\texttt{SolverTests.cs} & Operator Data Structure 
		(\misref{Module_operatordatastructure}), Solver 
		(\misref{Module_rangesolver}) & 17 & 100\% \\ \hline
		
		Output & \texttt{OutputTests.cs} &Output (\misref{Module_output}) & 6 & 
		100\% \\ \hline
	\end{longtable}
\end{center}

\section{Changes Due to Testing}
The test suites resulted in 

\section{Automated Testing}
		
\section{Trace to Requirements}
		
\section{Trace to Modules}		

\section{Code Coverage Metrics}


\end{document}